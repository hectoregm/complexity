\documentclass{article}
\usepackage[left=2cm,right=2cm,top=3cm,bottom=3cm,letterpaper]{geometry}
\usepackage[spanish]{babel}
\usepackage[utf8]{inputenc}

\usepackage{verbatim, array}
\usepackage{hyperref}
\usepackage{amsmath, amsfonts, amssymb}
\usepackage{graphicx}
\usepackage[T1]{fontenc}

\usepackage{amsthm}
\newtheorem{theorem}{Teorema}


\newcommand{\jimage}[2]{\includegraphics[width=#1\textwidth]{#2}\vskip10pt}
\newcommand{\jcimage}[2]{\begin{center}\includegraphics[width=#1\textwidth]{#2}\end{center}\vskip10pt}

\author{Héctor Enrique Gómez Morales}
\title{An ant-based algorithm for coloring graphs - Bui, Nguyen, Patel, Phan}
\begin{document}
\maketitle
\section{Introducción}
Al inicio del articulo se tiene la definición de varias conceptos básicos para el problema:

\begin{itemize}
  \item \textbf{coloración propia (por vértices)} es la asignación de los vértices de una gráfica con colores tales que cualesquiera dos vértices adyacentes tengan colores diferentes.
  \item \textbf{k-coloracion} cuando una coloración usa a los mas $k$ colores.
  \item \textbf{numero cromático} el menor numero de colores necesarios para colorear una gráfica, se denota como $\chi(G)$.
  \item \textbf{colisión} cuando dos vértices adyacentes tienen el mismo color.
\end{itemize}

El problema de coloración (graph coloring problem) es el problema de encontrar el numero cromático de una gráfica $G$. Se sabe que este problema es \texttt{NP-hard}, en particular se sabe que el problema de determinar si una gráfica puede ser coloreada con tres colores es un problema \texttt{NP-C}.

Dado que es un problema $NP-C$ y los algoritmos de aproximación que se tienen no son muy buenos se ha realizado mucho trabajo en el diseño de heurísticas para tratar de resolver el problema.

En el articulo de Bui se pone a consideración una heurística basada en hormigas para resolver el problema de coloración (ant-based algorithm for coloring graph \textbf{ABAC})

\section{Algoritmo ABAC}
La idea principal de \textbf{ABAC} que postula Bui es que se utiliza un conjunto de hormigas que se distribuyen aleatoriamente sobre la gráfica, cada hormiga sigue una lista de reglas para colorear su parte de la gráfica. Es decir a diferencia de otras propuestas cada hormiga solo colorea una porción de la gráfica y no toda la gráfica completa. Bui dice que esto presenta la ventaja de permitir una implementación distribuida, tengo curiosidad de que tanto se gana haciendo esta distribución puesto que se tiene la gráfica completa como estructura de datos global compartida, aunque haciendo uso de mutex o semáforos por vértice cuando se esta cambiando la coloración de un vértice permite una eficiencia mayor.

El algoritmo se puede reducir a lo siguiente:
\begin{enumerate}
  \item Se obtiene una \texttt{k-coloracion propia}, Bui recomienda su modificación del algoritmo \texttt{XRLF} pero en realidad es solo necesario obtener una coloración propia.
  \item De la k-coloracion propia obtenida en el paso anterior se induce una coloración con un numero $l$ de colores, con $l leqslant k$. Esta coloración puede ser no propia, es decir puede haber colisiones en la gráfica.
  \item Se empieza $n$ ciclos, en cada ciclo cada hormiga intenta colorear su porción de la gráfica usando el conjunto de colores definido anteriormente.
  \item Si al final del ciclo se tiene una coloración propia, es decir no hay colisiones entonces se reduce en uno el numero de colores disponible, en caso contrario se incrementa el numero de colores disponibles. Si han pasado $n$ ciclos o han pasado varios ciclos sin cambios se realiza un \texttt{jolt} o se detiene el algoritmo, sino se va al punto 3.
\end{enumerate}

En un algoritmo típico de hormigas (ant colony optimization \textbf{ACO}) se hace uso del concepto de feromona que es usada por las hormigas para decidir su acción a realizar, se puede decir que la feromona actúa como una memoria que permite a las hormigas el construir mejores soluciones, el algoritmo propuesto por Bui a primera vista no es un algoritmo \textbf{ACO} típico puesto que en ningún paso del algoritmo se hace mención de feronoma o de una estructura. parecida.

\section{Detalles críticos}

\subsection{Jolt}
Para manejar el caso en que se caiga en un máximo local, se hace una perturbación a la coloración actual de la gráfica por un método que Bui denomina \textbf{jolt}. Esta perturbación se realiza cuando no ha habido una reducción en el numero de colores durante los últimos $nJoltCycles = max \{n\ /\ 2600\}$, con $n$ el numero de vértices en la gráfica.
Para la perturbación se seleccionan el 10\% de los vértices de la gráfica con mayor numero de colisiones después se re-colorea los vértices vecinos de estos vértices seleccionados pero usando solamente el 80\% de los colores disponibles. Lo que se trata de hacer es provocar las suficientes perturbaciones en la coloración para hacer que esta salga del máximo local pero sin destruir totalmente la coloración actual.

\subsection{Movimientos de las hormigas}

Las hormigas se reparten aleatoriamente sobre los vértices de la gráfica. Cuando una hormiga esta en un vértice su objetivo es colorear el vértice de tal forma en que el numero de conflictos para ese vértice sea cero, si es posible. Si no es posible, entonces la hormiga elige el menor numero de color de la lista de colores disponibles de tal forma que se minimice el conflicto en el vértice, ademas trata de usar un color que no haya usado anteriormente. Después de colorear su vértice en que se encuentra la hormiga, esta se mueve a otro vértice y trata de colorearlo, se hacen $nMoves$ movimientos por cada hormiga. La hormiga toma un camino de longitud dos para llegar a cada nuevo vértice. La primera arista de este camino es elegido aleatoriamente, para la segunda arista se elige de tal forma que la hormiga caiga en el vértice con el numero de conflictos máximo de entre el conjunto de vértices adyacentes del vértice final de la primera arista. Adicionalmente cada hormiga tiene una listas tabú que contiene los vértices visitados recientemente para evitar revisitarlos durante el mismo ciclo.

La combinación de hacer movimientos de distancia dos, el elegir el vértice final del camino con el mayor numero de conflictos y el uso de una lista tabú de vértice visitados es la sustitución del uso de feromona de un algoritmo {ACO} típico, puesto ayuda a minimizar las colisiones en la gráfica llevando cuenta del trabajo ya realizado (colisiones, vértices ya visitados, colores no usados).

\end{document}
